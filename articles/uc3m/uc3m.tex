\documentclass[a4paper]{article}
\usepackage[utf8]{inputenc}
\usepackage{graphicx}
\usepackage{twocolpceurws}
\usepackage{xcolor}
\usepackage{url}

\def\infinity{\rotatebox{90}{8}}

\title{DameGender: Towards an international and free dataset about name, gender and frequency}

\author{
David Arroyo Menéndez, Madrid, Spain \\ \{davidam@gmail\}.com
}

\newif\ifdraft
\drafttrue
%\draftfalse
\input{macros}

\institution{DAMEGENDER}

\begin{document}

\maketitle

\begin{abstract}
  %% Introduction

  Equality of gender is the fifth objective of sustainable development
  for United
  Nations\footnote{https://www.un.org/sustainabledevelopment/gender-equality/}.

  This equality can be reached by measuring and analyzing data
  and making good politics with the results. Many gender studies
  count males and females based on their names, for
  instance, research papers, job positions, streets, ... The
  traditional research method is to use commercial APIs with
  proprietary data without idea about how the data was collected.
  Data may also be gathered from Wikipedia, lingüistic studies,
  scientific sites, or statistical offices.

  %% Methodology

  This approach is based collecting Open Datasets regarding name,
  gender and frequency from many statistical institutions. So, we
  need a scientific discussion about unifying formats and processing
  data easily.

  Therefore, Damegender (Free and Open Source Software) to retrieve
  and make calculus with these data.

  %% Results

  The dataset we used covers more than 20 countries in the occidental
  world encompassing many names with an accuracy of approximately 90%
  with it. This will create to measure gender gap to students and
  academics interested on the phenomenon without costs and on a
  reproducible way and more people will be contributing to fix the
  gender gap.

  %% Conclusion

  Free software and the data provided by statistical institutions make
  it possible to produce reproducible research for peer review. Thus,
  semantics and diversity can be more easily addressed.
  
\end{abstract}

\section{Introduction}
The United Nations has a goal to address the gender
gap\footnote{https://www.un.org/sustainabledevelopment/gender-equality/}.
You must remember that ``if you cannot measure it, you cannot
improve it"~\cite{thompson1833electrical} and ``Software
Engineering Economics is an invaluable guide to determining
software costs, applying the fundamental concepts of microeconomics
to software engineering~\cite{barry1981software}''. Free software and open
data lead to a reduction in costs, for example, many people and
institutions is using LibreOffice and Ubuntu (GNU/Linux) to avoid paying
the fees with similar products such as Microsoft Windows and Microsoft
Office. Gender detection tools based on the user name is based on API
solutions, providing a free software and open data solution. This will
createit competition in a market without a very strong leader, avoiding
payments and strategizing profits from a trademark, such as, Firefox or
Chrome.

Through the use of personal names, one may infer gender on
academical papers, books, newspapers and many interactions on Internet.
So, detecting gender from the names may be a strategical way to
measure gender gap.

Many users today are using APIs such as Genderapi, Genderize,
Namsor, or NameApi, Wikipedia, or Free Software solutions
(NLTK\cite{loper2002nltk}, R Gender, Gender Detector and Gender
Computer\footnote{https://github.com/tue-mdse/genderComputer}.

Traditional open source solutions has a few number of names due to
use files of a single country or being software not maintained in
the long time. And Wikipedia is storing few names per country.

However, the gender gap is a problem recognised in United Nations and
the IT market is leading big inequalities in economic and gender gap.
This article presents data collected to assist in finding the solution
to a number of problems (search engine, infering gender in csv files,
names in different countries, wide dataset) faced by
the industry as well as other problems not solved in an industrial way
such as counting males and females in GitHub repositories, mailing lists,
...

A previous study~\cite{karimi2016inferring} dicussed these datasets as
a way to improve their accuracy, comparing tools that use different
public datasets (SSA~\footnote{https://www.ssa.gov/oact/babynames/limits.html},
IPUMS~\footnote{https://usa.ipums.org/usa-action/variables/NAMEFRST},
namdict~\footnote{https://raw.githubusercontent.com/lead-ratings/gender-guesser/master/gender\_guesser/data/nam\_dict.txt}, etc)

So, a goal is to augment the number of names using official statistics
and taking into account diversity goals such as non binary gender and
cultural minorities.

With DameGender we will make science reproducible\cite{peng2011reproducible}
in fields with similar works such as Natural Language Processing
(gender detection from the name~\cite{sun2019mitigating}), social
sciences or journalism (gender
gap~\cite{holman2018gender,mislove2011understanding,niemi2017gendered,de2014genero}),
linguistic~\cite{hutson2016gender,van2020gender,okal2018linguistic},
software engineering~\cite{vasilescu2012gender}, among other fields.

The remainder of this article is structured as follows:

Section~\ref{sec:stateofart} presents the main research measuring the
gender gap and gender detection tools using name.

Section~\ref{sec:design} gives vocabulary and philosophy about to
choose sources and to face the diversity troubles building a dataset.

Section ~\ref{sec:measuring} explains an application for this
dataset: to measure gender gap in GNU/Linux.

Section~\ref{sec:conclusions} points a summary about this approach and
future works.

The contributions of this article are:

1. An integrated solution in the different applications field relative
to inferring gender from the name.

2. A collection of open datasets retrieved from statistical sources
and standardized in an unique format.

3. A new study applying DameGender to count males and females in
GNU/Linux.

4. An approach based on reproducible results.


\section{State of the Art}
\label{sec:stateofart}

\subsection{About Gender Gap}

To reduce gender gap refers to equality between males and females,
and non discrimination policies. Gender refers to the sex of a person
determined in the moment of the birth, although it can be changed
throughout life. Discussions about gender definitions refers to these
problems. However, there is a consensus determining gender, frequency
and names with official statistics released by the institutions in the
states.

Measuring the gender gap requires set indicators. ~\cite{world2021global}
has been proposed economy, health, education and politics. And in
United Nations\footnote{https://www.unwomen.org/} there are indicators
used to measure disparities such as laws, education, maternal mortality,
political participation, poverty, domestic work, gender parity in
the work, to access to the economy, youth issues (access to
studies and/or work), violence against women, climate justice,
access to the justice, health, ...

It's possible to make impactful decisions on an issue through research
results that have taken these indicators into consideration. For example,
~\cite{miyake2010reducing} concluded that making affirmations about
ethical values reduced the gender achievement gap in colleges.

Measuring the gender gap in social research, such as the survey. For
example, in ~\cite{bimber2000measuring} presented two factors affecting
the gender gap on the Internet (access and use) by socioeconomic and gender
reasons in a survey that collect data over several years. Internet
access is vital to today's in the economy, education.
~\cite{10.1007/978-3-319-39225-7_13} is using a survey of 2000
contributors.

\subsection{Counting males and females on the Internet. Why? Where?}

This work focused on retrieving data from secondary sources such as
GitHub, Wikipedia, APIs, websites in general, mailing lists, etc.
Previous research works about factors modifying several gender gap
indicators (economy, education, politics) were obtained from secondary
sources.

For example, a social scientist studying gender gap in
journalism~\cite{alvarez2012journalism} can count males and
females on Twitter. These metrics are important because the journalism
is evaluating gender gap in political, education, or the economy,
... Meanwhile, Computer Science making research about how to
count males and females in Twitter~\cite{burger2011discriminating}. In
these studies the name, nickname, photo, and identifying gender are
retrieved from these data.

~\cite{burger2011discriminating} presented several configurations of a
language-independent classifier for predicting the gender of Twitter
users. The large dataset used for the construction and evaluation of
these classifiers was drawn from Twitter users who also completed blog
profile pages.

~\cite{mislove2011understanding} analyzed the Twitter population,
including the gender. The gender was inferred making queries
from the names to the dataset provided by the United States Census
Bureau.

~\cite{wagner2015s} analyzed the gender gap in Wikipedia,
showing evidence of more subtle forms of gender inequality explaining
how to solve these evidences. To measure gender inequality has been
developed the next bias: coverage, structural, lexical (ex:
discriminatory words for women), and visibility.

Computer Science is generating many Forbes billionaires and the public
code may help to understand the gender gap in this field, which may
have some importance to the economy. Public repositories can
be used to build indicators about the economy in Computer Science with
more factors, such as job positions, value of companies, etc.
~\cite{zacchiroli2020gender} conducted the first large-scale
longitudinal study of gender imbalance among authors of
collaboratively developed, publicly available code, where
contributions by female authors remain scarce less that 8 \% of
commits was able to be detected were from women, confirming decades of
gender imbalance in Free/Open Source Software (FOSS). Steffano used
to namdict~\footnote{https://raw.githubusercontent.com/lead-ratings/gender-guesser/master/gender\_guesser/data/nam\_dict.txt} dataset with genderguesser to infer gender from
the name. ~\cite{vasilescu2015gender} determined that women programmers
are in the minority in OSS and other technical fields, although increased
gender and tenure diversity is associated with greater productivity.

~\cite{vasilescu2012gender} explored the popular Q\&A about
technological issues called StackOverflow, which summarizes that the
percentage of women engaged in SO is greatly imbalanced, and men
represent the vast majority of contributors.

~\cite{izquierdo2018openstack} revealed that data few females
contribute code or take political responsability in the OpenStack community.

Related to the gender gap in science, ~\cite{holman2018gender} presented a
code in R using genderize API and provides a good approach about how to
calculate gender gap inferring gender from an author names retrieved from
arXiv.

\subsection{Automatic approaches to infer gender}

There are several ways to infer gender from Internet
sources: hand written, images, documents and names.

~\cite{liwicki2011automatic} presents a method inferring gender from
hand written texts with a 67.5 \% accuracy.

~\cite{gallagher2008estimating} combines image based gender and age
classifiers with the cultural information provided by first
names to recognize people with no labeled examples with results near
to 60 \% accurate.

~\cite{argamon2003gender} explains that females use many more
pronouns, while males use many more noun specifiers, in a large subset of
the British National Corpus covering a range of genres. Therefore,
~\cite{koppel2002automatically} presented a document classification
system with accuracy of approximately 80 \%. ~\cite{cheng2011author}
exposes a feature selection and a model built using machine learning
resulting in 85.1 \% accurate rate for identifying gender from text.

\subsection{Infering gender from name}

The tools used to infer gender from a name are tipically based on
datasets that, at a minimum, is include gender and name as minimum.

~\cite{liu2013s} presented a method to infer gender from
first names in Twitter, the dataset was hand coded by agreement
between three Amazon workers with 50,000 Twitter users select at
random with only 12,681 gender labels. The goal of this study was
to determine the incremental value of using the user name as a feature
in gender inference based on tweets.

~\cite{mueller2016gender} presented how to infer gender in
Twitter. They used namdict and the United States census as datasets. The
features were 'number of consonants', 'number of vowels', 'number of
syllables', 'number of bouba consonants', 'number of bouba vowels',
'number of kiki consonants', 'number of kiki vowels'. The
classification model was created using SVM.

\subsection{Related ideas}

~\cite{ambekar2009name} presented a system to classify name
and ethnicity from open sources using machine learning to
extract a name list from Wikipedia. A more recent work is
~\cite{nadri2021relationship}, in which presented NamPrism was
applied to massive software repositories.

~\cite{bollegala2010automatic} presented another approach that
used a lexical-pattern-based approach to extract aliases of a
given name, with a set of names and their aliases as training
data to extract lexical patterns. The candidates are ranked
using various ranking scores. Support vector machines were
used to construct the ranking function.

\subsection{Related Standards}

ISO/IEC 5218 proposes the following norm about coding
gender: ``0 as not know'', ``1 as male'', ``2 as female''
and ``9 as not applicable''.

The RFC 6350
(vCard)~\footnote{https://datatracker.ietf.org/doc/html/rfc6350}
has these categories: ``m as male'', ``f as
female'', ``o as other'', ``n as not applicable'' and ``u as
undefined''. Based on this standard, those conducting web
publishing can use CSS classes using a web standard such as
h-card~\footnote{https://github.com/microformats/h-card}
microformats in the context of to write forms in web interfaces
consider w3 lectures~\footnote{https://www.w3.org/International/questions/qa-personal-names}

\subsection{Summary}

The first name of a subject is the is the key factor used to
determine gender in the State of Art gender inference tool.
However, in many contexts there are more features: surnames,
text, images, nicknames, ... The first name can be useful to
infer another stuff such as race, ethnicity or culture, too.

Machine learning and the previous features selection is being
used in many works, although there is an open discussing as
to which is the best approach

The datasets can be built by human experts, although there are
some open datasets used several times in these researches, such as
namdict, or the United States census.

\section{Design}
\label{sec:design}

\subsection{Truth and falsehood in names, gender, and frequency}
\label{sec:truthandfalsehood}

The current idea in the field accepts that using name, gender and
frequency is ok because there are people paying for or downloading
a product. Typically, this is an acceptable assumption, although
the consumer may purchase a bad product due to a good marketing
strategy, a monopoly or there is a fraud, ... Consumers may also 
trust in the government statistics regarding the economy,
demography, or democracy. Therefor the people may trust the data
for names, gender, and frequency. In Damegender, we are trusting
in: the market's point of view and the official statistic's
point of view.

Sometimes there are problems downloading official statistics, but
there are people who have retrieved these data with web scraping.
These files with another idea about truth.

Another problem arises when the government changes the data,
sometimes communicating it to the users and other times
not. This may be problematic for upgrades, but not with the truth,
since changes can be traced. 

With an international free dataset of names, gender, and frequency,
we can build reproducible science in fields such as natural language
processing (gender detection from the name), the social sciences, 
journalism (gender
gap~\cite{holman2018gender,mislove2011understanding,niemi2017gendered,de2014genero}),
linguistics~\cite{lawson2005russian,krueger1962mongolian,van2020gender,agyekum2006sociolinguistic,fraser1987lexicon},
or software engineering~\cite{vasilescu2012gender}.

\subsection{Gender, language, nation and diversity}
\label{sec:diversity}

There are rules and exceptions in different languages to predict if
a name is about male or female when you don't know the name. For example,
in Spanish or English, there are more names ending with 'a' classified as
females than classified as males. However, Andrea is female in Spain and
male in Italy. So, it is useful to understand the language and culture
associated with a name. Language is close to nation, but there are
differences, for example, in Spain there are several languages Basque,
Catalan, Castillian. Spanish is the main language in Spain.
and in other countries such as Argentina, Mexico, Ecuador and Bolivia,
Thus, it is helpful to know the language and nation a name or surname
has originated from to help to detect gender.

Some countries, such as Spain, are providing free datasets for
surnames but we need more efforts from many countries on this
objective. However, Wikipedia and machine learning are working
to relate names and surnames with ethnicity~\cite{ambekar2009name}.

\subsection{Damegender open datasets collection}
\label{sec:damegender}

In Damegender, we unified the different formats for name, gender,
and frequency from statistical offices of the following countries:
Argentina, Austria, Australia, Belgium, Canada, Switzerland, Germany,
Denmark, Spain, Finland, France, Great Britain, Ireland, Iceland,
Norway, New Zealand, Mexico, Portugal, Russia, Slovenia, Sweden,
the United States of America and Uruguay.

It has been applied the criteria that one person usign a name is
a gender vote (male or female)\footnote{On the future, laws about
non binary could be including other options, but only male and
female was found as valid options in the open datasets when this
article is being written}. So, for each country has been choosen
births or total of people using names in the country. Later, it
has been merged these datasets building a free and international
dataset. 

Damegender uses surnames given by statistical institutions (Spain,
Russia, the United States of America and Argentina in this moment).
However, there are few statistical institutions that provide
surnames and the lingüistic diversity is a good point, has been
added surnames for all countries from Wikidata. Perhaps in the future
Wikidata will become the best source of data for names and surnames,
but now there are few elements.

When the work is finished, we could to rebuild machine learning models
to predict new names and nicknames in any language and culture. The
results is the longest list of public names with a scientific approach.

\begin{table}[t]
\footnotesize
\begin{tabular}[]{lcccc}
  \hline
  Dataset & SSA & namdict & NLTK & Damegender \tabularnewline
  males & 91.320 & 48.821 & 2.943 & 257.925 \tabularnewline
  females & 91.320 & 48.821 & 5.001 & 304.553 \tabularnewline
  \hline
\end{tabular}
\caption{Comparison of the number of names between open data solutions}
\label{table:DifferentNamesMeasures}
\end{table}

A possible criticism about this approach is the Leslie
Problem\cite{blevins2015jane}: the match between gender and name
depends of the year. The solution to this problem is to introduce the
age of the person in question. The most common use case states that
the input is the name and the output must be gender, frequency and
percentage. Therefore, a decision about gender must be made without
data on the age or surname in most cases. This dataset was designed for
the most of used use cases. We can take into account other inputs,
such as surname or age to improve the accuracy. There are many open
datasets with names and frequencies that have been classified by years.
Therefore, this problem can be fixed with open data, too.

\begin{table}[t]
\footnotesize
\begin{tabular}[]{lcccc}
  \hline
  Dataset  & Accuracy & Precision & Recall & F1-Score  \tabularnewline
  Damegender &  0.8756  & 0.9638    & 1.0    & 0.925  \tabularnewline
  \hline
\end{tabular}
\caption{Several precision measures about the DameGender international dataset}
\label{table:DifferentAccuracyMeasures}
\end{table}

We have measured about the international DameGender dataset,
using the dataset explained in
\cite{10.7717/peerj-cs.156} to reach accuracy (0.8756), precision
(0.9638), recall (1.0) and f1-score (0.925). With other test datasets,
similar results:

\begin{itemize}
\item Females scientists in Wikipedia (accuracy: 0.89)
\item Males scientists in Wikipedia (accuracy: 0.98)
\item FIFA football dataset (accuracy: 0.93)
\item Scientific test dataset designed\cite{sebo2021performance} (accuracy: 0.88)
\end{itemize}


\subsection{Free APIs for free datasets?}
\label{sec:freeapis}

There area number of open data websites that allow the user to retrieve structured open data without cost. These sites include Wikipedia with SPARQL and OpenStreetMap with API rest, among others.

The open datasets for names, gender, and frequency are being modified once a year, at most for each statistical institution.

DameGender contains python scripts designed to create the
different datasets and publish json files that could be used as a free
API rest publishing the json files in sites as GitHub pages, GitLab
pages, or similar sites with free uploads.

\begin{verbatim}
  $ cat DAVID_all.json
  [{
      "name": "DAVID",
      "frequency": 4856689,
      "males": "99.73267796229078 %",
      "females": "0.26732203770922947 %"
  }]
\end{verbatim}

Therefore, it may be possible to have free API rest about names,
gender, and frequency with reduced costs to fix the gender gap
in a collaborative way similar to Wikipedia, OpenStreetMap or
many free software projects.

\section{Measuring gender gap. GNU/Linux as use case}
\label{sec:measuring}

With a trust open dataset for names, gender and frequency it is
easy to measure the gender gap. Students and academics could
measure the gender gap inexpensively and meet the fifth Sustainable
Development Objective of the United Nations, which is to erase
the gender gap completely.

This section is divided into counting males and females using
Debian, GNU, and Linux.

We obtained the csv files using different methods to determine
the names about the people in these communities.

In the Debian community all members must collaborate with
a gpg key, so we can count males and females from the keyring.
The keyring was imported with gpg commands and later 
the keyring was placed in a csv file.

GNU\footnote{https://www.gnu.org/people/} and
Linux\footnote{https://www.kernel.org/doc/html/latest/process/maintainers}
have collaborate websites for these projects. So, making web
scraping scripts we have downloaded the people and processed
the people to csv files.

In DameGender, has been developed csv2gender, a software with a csv
file as input and deploy a statistics graph and/or return the result
of males, females and unknowns about the input.

To make easy to reproduce the experiment we are pasting the commands
used with the version 0.3.4 of Damegender.

\begin{verbatim}
python3 csv2gender.py files/gnu.csv
 --first_name_position=0
 --title="GNU maintainers grouped by gender"
 --dataset="inter"
 --outcsv="files/gnu.gender.csv"
 --outimg="files/gnu.gender.png"
 --noshow --delete_duplicated

python3 csv2gender.py files/linux.csv
 --first_name_position=0
 --title="Linux maintaners grouped by gender"
 --dataset="inter"
 --outcsv="files/linux.gender.csv"
 --outimg="files/linux.gender.png"
 --noshow --delete_duplicated

python3 csv2gender.py files/debian.csv
 --first_name_position=0
 --title="Debian maintaners grouped by gender"
 --dataset="inter"
 --outcsv="files/debian.gender.csv"
 --outimg="files/debian.gender.png"
 --noshow --delete_duplicated
\end{verbatim}

\begin{figure}
  \centering
  \includegraphics[width=0.6\textwidth]{images/debian-gnu-linux.pdf}
  \caption[Caption for LOF]{Males (blue), Females (orange) and Unknows (green) in Debian, GNU and Linux}
\end{figure}

The inter dataset was created merging several open datasets downloaded
from official statistics sites from different nations: Austria,
Australia, Belgium, Canada, Germany, Denmark, Spain, Finland, Ireland,
Iceland, Mexico, New Zealand, Portugal, Slovenia, United States of
America, Uruguay and France. That's a good representation of the
Western World and the Free Software world is populating this world's
area\cite{gonzalez2008geographic}.

Linux divides the developers in 537 males (73.9\%), 98 females
(13.5\%) and 92 unknowns (12.7\%). The number of unknowns is due to
different reasons, but it's so common in Linux that the developer is a
company and not a name of a person.

GNU divides the developers in 164 males (89.6\%), 12 females (6.6\%)
and 7 unknowns (3.8\%)

The GNU people has a number lowest in females, they are the founder of
the Free Software philosophy, the Debian principles and the Open
Source philosophy was invented later influenced by GNU with very
similar practical decisions (for example: deciding licenses for the
software). Richard Stallman returned to be president recently
apologizing by his personal behaviour with the
females.\footnote{https://www.fsf.org/news/rms-addresses-the-free-software-community}

Debian is a distribution, the project who makes the CD/DVD and the
software ready to be downloaded from Internet with the
dependencies. There are many distributions, such as, Ubuntu or RedHat
so it is not representative, but it's interesting to understand that
the numbers are similar in Debian dividing the developers in 408 males
(75\%), 69 females (12.7\%) and 67 unknowns (12.3\%).

%% \includegraphics[width=0.9\textwidth]{images/debian.gender.pdf}

\section{Conclusions and Future Works}
\label{sec:conclusions}

This paper is explaining the application about Damegender, the
motivations (reproducible research, fix gender gap to reach an
objective of United Nations, fields of application: linguistic, social
sciences, software engineering, natural language processing,
journalism, ...)

A good improvement is to build an international, universal and free
dataset about names, gender and frequency with the right design with
the current state of the job, attending to the diversity (LGBT
options, cultural minorities, ...).

This paper has explained what technologies is involved on reduce costs
about gender gap (gender detection from the names, api rest, semantic
web, ...)

Augmenting the number of countries with statistical institutions
giving names, gender and frequencies with Open Data will be augmenting
the accuracies and giving more attention to the diversity.

The current state of work is the longest Open Dataset about names,
gender and frequency with more than 20 countries representing the
Western World, being a solution with low number of unknowns in the real
world.

The future works is about changes in the big software industry.

Making searches with strings about personal names (ex: Leticia) in
search engines such as Google, these strings are not being classified
as personal names, one solution will be data structured such as
JSON-LD, microdata, microformats, rdfa ... Another solution will be
store in the servers the Open Data Collection about names, gender and
frequency and identify the context about the string is a personal
name, that's an easy problem in popular sites such as Wikipedia,
academic websites, ...

If the search engine identify the string as personal name, it can help
to the user about the gender. That is similar than other problems such
as streets, products, ... where you are giving additional information
such as maps in streets or prices in products.

Other sites such as Github or Gitlab could be giving data about
gender of developers in the site or in the software project with these
datasets.

Another industry is about match sites (Meetic, Tinder, ...) where only
is important photos, age and gender generally. It could be possible to give
to users gender, photo and interests from personal names our open data
collection and information related in Internet.

%% These data reveals that the situation has been improved with respect
%% the long time. In \cite{10.1007/978-3-319-39225-7_13} speaks about a
%% female participation of around 2 {\%} to 5 {\%}.

\section*{Acknowledgments}

We would like to thank: Statistical institutions by release Open
Datasets about names, gender and frequency. Luz Galvis by the
software contributions, Daniel Izquierdo and Laura Arjona for
starting this research field at URJC all those working with Jesús
González Barahona and Gregorio Robles. 

\bibliographystyle{alpha}
\bibliography{uc3m}

\end{document}
